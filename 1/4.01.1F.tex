\subsection{MongoDB}
Für die Abbildung des Anwendungsszenarios wurde sich dafür entschieden die Datensätze, bis auf kleine Korrekturen, unverändert zu übernehmen. Dafür wurde jeder Datensatz, jede Zeile der CSV-Datei, als ein Dokument importiert. Um schnelle Abfragen auf die Daten zu ermöglichen wurden häufig genutzte Attribute indeziert. 

TODO: INDEZIERUNG?! Wer kümmert sich darum?! 

\subsubsection{Import der Daten}
Die Beschreibung des Datenimports ist mit Hilfe des ETL-Prozesses strukturiert: 

\emph{Extraktion} - 
Da die Daten in Form einer CSV-Datei bereit gestellt wurden, entfällt dieser Schritt. Allerdings müssen zwei kleine Anpassungen an der CSV-Datei vorgenommen werden: 

\begin{itemize}
\item \textit{Umlaute korrigieren} - Die Umlaute in der CSV-Datei wurden, vermutlich durch den Export aus dem Quellsystem, fehlerhaft dargestellt. Diese wurde so ersetzt, dass sie wieder menschenlesbar sind.
\item \textit{Kopfzeile anpassen} - Die Werte in der Kopfzeile listen die Attributnamen auf. Diese werden später die Schlüssel der Werte des Dokuments in der Datenbank sein. Schlüsselnamen dürfen weder Leerzeichen noch Umlaute enthalten. Somit wurden die Leerzeichen entfernt und durch Unterstriche ,,\_`` ersetzt sowie alle Umlaute mit einfachen Vokalen dargestellt. So wurde beispielsweise aus ,,\"a`` ae oder aus ,,\ss{}`` ss. 
\end{itemize}

\emph{Transformation} - 
Da MongoDB Imports primär über JSON erwartet, musste die CSV-Datei in das JSON-Format transformiert werden. Die Transformation der Daten wurde mit Hilfe eines Online Tools durchgeführt. Diese ist unter der URL \url{http://http://www.csvjson.com/csv2json} zu finden. 

\emph{Laden} - 
Der Import in die MongoDB erfolgt über die MongoDB-Konsole. Um die Arbeit hier etwas zu vereinfachen wurde das GUI-Tool ,,RoboMongo`` verwendet. Die Syntax des Import-Befehls ist Listing \ref{lst:mongodb_import} zu entnehmen. Dabei können mehrere Dokumente in einem Array per Komma getrennt gleichzeitig importiert werden. \\

%https://en.wikibooks.org/wiki/LaTeX/Source_Code_Listings
\begin{lstlisting}[caption={MongoDB Importfunktion},language=java,captionpos=t,numbers=none, numberstyle=\tiny,breaklines=true]
db.getCollection('Collection_Name').insert([{Document01}, 
 	{Document02},
 	{"Attribut01":"Wert01", "Attribut02":"Wert02"}])
\end{lstlisting}\label{lst:mongodb_import}

Allerdings sind die 1639 Datensätze mit 32,7 MB Gesamtgröße zu groß für einen Import. Das Maximum hier liegt bei circa 16 MB. Also wurden die Datensätze in drei Teile aufgeteilt und nacheinander importiert. 

TODO Bild: 

\subsubsection{Anwendungsszenario}
TODO Parametrisierbarkeit 

Die erstellte Website soll von Professoren und Studenten dafür genutzt werden, um nach Informationen über alle bisherigen Abschlussarbeiten suchen zu können. Dies kann beispielsweise für Professoren statistische oder Auswertungszwecke verfolgen oder für Studenten, welche auf der Suche nach einem Abschlussarbeitsthema sind, als Informationsquelle dienen. 

REWORK

Im folgenden werden acht erstellte Use Cases vorgestellt. Darunter sind vier, die für Studenten, zwei die für Professoren und zwei, welche für beide Stakeholder zugeschnitten sind. Des Weiteren ist Use Case 8 ein allgemeiner Use Case, welcher auf alle vorgestellten Persistenzsysteme gleichermaßen angewandt wurde um als Performance-Vergleich zu dienen. Die Use Cases bestehen aus einer kurzen textuellen Beschreibung, einer Definition von Output und wahlweise Input sowie dem Query für die MongoDB-Konsole. Die Querries sind Anhang \ref{use_cases} zu entnehmen. 

REWORK

die Implementierung programmiertechnisch unterschiedliche Anfragen vorstellen. 
Eine Reihe ähnlich gearteter Selects ist da sicher nicht so hilfreich. 

\subsubsection{Oberflächengestaltung}
Art der Ein-/Ausgabemasken 

\subsubsection{Implementierungskonzept}
Vorgehensweise, Framework, Systemarchitektur 

\subsubsection{Schnittstelle der Anwendung zur Datenbank}
verwendete Kopplung / Schnittstelle Anwendung zur Datenbank 

\subsubsection{Test der Anwendng}
Funktionieren die UseCases? \\
Performanceaspekte
