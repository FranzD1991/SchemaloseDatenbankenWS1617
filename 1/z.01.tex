\begin{abstract}
\section*{Zusammenfassung}\markboth{Zusammenfassung}{}
  \addcontentsline{toc}{chapter}{Zusammenfassung}Moderne Datenbanken sehen sich mit diversen Problemen konfrontiert. Es sollen extrem große Datenmengen verarbeitet werden und die Antwortzeiten sollen niedrig gehalten werden. Ferner soll es für sehr viele Nutzer gleichzeitig möglich sein Abfragen zu stellen. Aus diesen Gründen ist es in modernen Datenbanken notwendig ist, Datenstrukturen möglichst flexibel und performant zu speichern. Diese Probleme sollen mit verschiedenen Ansätzen aus dem Umfeld der NoSQL Datenbanksysteme gelöst werden. Die dokumentenbasierte Speicherung versucht für Probleme, welche sich meist auf große Mengen mit heterogenen Daten beziehen, einen Lösungsansatz zu finden. In diesem Umfeld ist die Funktionsweise der MongoDB äußerst populär, da sie auf hohe Leistung, große Datenmengen, hohe Flexibilität und einfache Skalierbarkeit ausgelegt ist. Die Grundlage für die der MongoDB zugrunde liegenden dokumentenbasierte Speicherung bieten die Key-Value Stores, welche unformatierte Werte unter einem Key persistieren. Document Stores vereinen den Vorteil der Schemafreiheit der Schlüssel-Wert Zuordnungen mit der Möglichkeit zur Strukturierung von Daten. Dadurch sind Datenbanken wie MongoDB in der Lage komplexere JSON ähnliche Datentypen zu verarbeiten und basierend auf ihren Attributen zu indexieren. Die hier umrissenen theoretischen Grundlagen werden detailierter beschrieben und in dem NoSQL Kontext bewertet.\\


Um die theoretischen Grundlagen zu verdeutlichen werden mithilfe eines Anwendungsszenarios die Vorteile von MongoDB verdeutlicht. Grundlage für das Anwendungsszenario ist das logische Schema einer relationalen Datenbank, in der sowohl Abschlussarbeitsthemen, als auch alle beteiligten Personen, wie beispielsweise Studenten oder Prüfer, sowie Organisationen, gespeichert sind. \\
Ziel des Anwendungsszenarios ist es, dieses Schema mit Hilfe von MongoDB umzusetzen. Ferner soll die Eignung von MongoDB in diesem Kontext analysiert und bewertet werden. Damit MongoDB in diesem Kontext bewertet werden kann, werden unter anderem Performancetests an der Datenbank durchgeführt. Weiter soll eine Web-Anwendung implementiert werden, welche die grundsätzlichen Datenbankoperationen an der MongoDB veranschaulicht.
\end{abstract}
