\section{Key-Value Stores}
\subsection{Einleitung}
Key-Value Stores oder Key-Value Datenbanken sind eine der einfachsten Formen der NoSQL Datenbanken. Sie können beliebige Werte (Englisch: Values) unter einem definierbaren Key persistieren und über diesen selektieren. Dabei können sie mit einem herkömmlichen relationalen Datenbanksystem (RDBS) verglichen werden in welchem die Tabellen nur zwei Attribute besitzen: eine ID- und eine Value-Spalte \cite{sadalage01}. Im Unterschied zu dieser relationalen Datenbank erwarten Key-Value Stores keinen Datentyp für den Value. Hier können beliebige Werte als blob gespeichert werden. Auch können Key-Value Datenbanken weder Relationen oder Beziehungen zwischen den Einträgen herstellen. Für die Semantik, Korrektheit der Werte, Beziehungen o.ä., ihre Verwendung, welcher Datentyp vorliegt und etwaiges Fehlerhandling ist allein die Anwendung, welche die Datenbank nutzt, verantwortlich. Dieses sehr grundlegende Verhalten soll eine möglichst hohe Performance und Skalierbarkeit ermöglichen. 

\subsection{Eigenschaften}
Einige Key-Value Stores gehen einen Schritt weiter und können das zu speichernde Aggregat differenzieren. So kann Redis beispielsweise Listen, Sets oder Hashes speichern und unterstützt einfache Operationen auf diesen Datenstrukturen. Somit verschwimmt in der Praxis der Unterschied zwischen Key-Value und Document Stores bei einigen Vertretern sehr.

Im Folgenden werden einige Eigenschaften von Key-Value Stores näher betrachtet und aufgezeigt wie diese sich von einem Standard RDBS unterscheiden.

\subsubsection{Konsistenz}
Da die Key-Value Datenbank Änderungen des Values nicht bestimmen kann, kann die Konsistenz somit nur über den Key und die damit verbundenen Operationen get, put oder delete gewährt werden \cite{sadalage01}. 

In verteilten System wird häufig die \textit{eventually consistent} Eigenschaft angewendet. 

\subsubsection{Transaktionen}
Für Transaktionen gibt es in der NoSQL Welt keine einheitliche Spezifikation. Jeder Anbieter von Key-Value Stores implementiert diese anders, Riak benutzt hier beispielsweise das Quorum-Konzept \cite{sadalage01}.

\subsubsection{Query-Eigenschaften}
Da sich Key-Value Datenbanken nicht für den gespeicherten Value interessieren kann man Queries nur auf den Key ausführen. Auf Grund dieses sehr einfachen Handlings sind Key-Value Stores nicht für Anwendungen geeignet, welche über den Inhalt filtern oder suchen müssen. Hier muss die Anwendung wissen welche Daten unter welchem Key zu finden sind und welchen Key sie aufrufen möchte. Da nicht immer der Fall eintritt, dass der Key zu einem Datum bekannt ist, stellen manche Key-Value Datenbanken allerdings eine Suchfunktionalität oder die Möglichkeit der Indexierung bereit, damit nicht jeder Datensatz von der Anwendung geladen und einzeln durchsucht werden muss \cite{sadalage01}. 

\subsubsection{Struktur der Daten}
Wie bereits erwähnt können in Key-Value Datenbanken verschiedenste Datentypen wie blob, text, JSON, XML oder weitere gespeichert werden, da die Datenbank sich um diese Werte nicht weiter kümmert. Bestimmte Datenbanken, wie beispielsweise Riak können einem Request einen Content-Type übergeben womit diese weiß wie sie die Daten behandeln soll \cite{sadalage01}.

\subsubsection{Skalierbarkeit}
Um die Performance zu erhöhen arbeiten die meisten Key-Value Stores mit „Sharding“. Hierbei entscheidet der Wert des Keys auf welchem Knoten bzw. Shard der Datenbank ein Datum gespeichert wird. So werden beispielsweise alle Keys, welche mit g beginnen auf einem anderen Shard gespeichert als solche, die mit a oder einem nummerischen Wert beginnen. Zusätzliche Shards können sehr einfach ergänzt werden. Um hierbei die Konsistenz nicht zu gefährden, sollte ein Knoten ausfallen, können diese Shards repliziert werden. Die einzelnen Key-Value Datenbanken bieten hierbei verschiedene Möglichkeiten an. So kann bei Riak angegeben werden, wie viele Knoten bei Schreib- und Lesezugriffen mindestens beteiligt sein müssen damit die Transaktion als erfolgreich gilt. 

\subsection{Abgrenzung zu Document-Stores}
Sehr ähnlich, Unterschied: Das Aggregat muss eine bestimmte Struktur haben/ von bestimmtem Typ sein
Mit Document Stores sind Abfragen auf Attribute möglich. Hier nur per Key.
In der Praxis verschwimmt die klare Trennung.
Unfertig …

\subsection{Anwendungsfälle}
Da die Keys in den Key-Value Stores von der Anwendung verwaltet werden, können diese verschiedenen Charakter haben. Eine Möglichkeit ist diese Keys durch einen bestimmten Algorithmus, zum Beispiel eine Hashfunktion, zu generieren. Man kann den Key auch durch Benutzerinformationen, wie Email-Adresse oder User-ID generieren. Alternativ kann dieser auch aus einem Zeitstempel oder anderen Attributen erstellt werden. Dies macht Key-Value Stores attraktiv um Session Informationen, Einkaufswagen im Online-Shopping, Benutzerprofile oder Protokolldateien abzuspeichern, da diese Daten unter einem eindeutigen Key als ein Objekt gespeichert werden \cite{sadalage01}. 

Anwendungsfälle in welchen Key-Value Stores weniger gut geeignet sind:
\begin{itemize}
\item bei Abhängigkeiten oder Beziehungen zwischen den Daten wie sie in herkömmlichen RDBS zu finden sind.
\item bei \textit{Multioperation Transactions}, in welchen bei einem Fehler die restlichen Operationen zurückgedreht werden müssten.
\item bei (Such-)Anfragen, welche den Inhalt der Daten betreffen.
\item bei Operationen, die sich auf mehrere Datensätze zum gleichen Zeitpunkt beziehen, da eine Operation in einem Key-Value Store auf einen Key beschränkt ist.
\end{itemize}

\subsection{Bewertung}
Key-Value Stores ziehen ihren großen Vorteil aus der Einfachheit ihrer Implementierung. Diese ermöglicht ihnen eine hohe Performance. Hierbei verzichten sie absichtlich auf Funktionen, wie Beziehungen, feste Datentypen und die Möglichkeit Abfragen auf jedes Attribut eines Datensatzes zu stellen, die durch die relationale Datenbankwelt bekannt geworden sind. 

\section{Document Stores}
\subsection{Einleitung}
Document-Stores speichern strukturierte Daten in Datensätzen, die Dokumente genannt werden. Grundsätzlich vereinigen sie die Schemafreiheit von Key-Value-Stores mit der Möglichkeit zur Strukturierung der gespeicherten Daten. Somit lassen sich Document-Stores als Erweiterung von Key-Value Stores betrachten. Vorteil hierbei ist, dass es bei Document-Stores im Gegensatz zu Key-Value Stores möglich ist, komplexere auf den Inhalt (die Attribute) der Dokumente gerichtete Abfragen zu stellen. Dies ist bei Key-Value-Stores nicht möglich.

\subsection{Eigenschaften}
Bei der Betrachtung von Document-Stores bezieht sich das Wort \grqq document\grqq auf die hierarchische Struktur der abgespeicherten Daten. Die hier betrachteten Dokumente bestehen jedoch lediglich aus binären Daten oder einfachem Text. Ferner kann es sich um Semi-strukturierte Daten wie JSON oder XML handeln. Während ältere Document-Stores auf XML setzten, ist bei neueren Document-Stores das JSON ähnliche BSON üblich. 
\\

In einer Document-Store basierten NoSQL-Datenbank ist es möglich beliebigen Daten zu speichern, ohne das die Datenbank im Vorfeld Wissen über die Struktur der Daten hat oder weiß was die Daten bedeuten. Gegenüber relationalen Datenbanken bietet dies den Vorteil, dass das Datenbankschema bei kleinen Änderungen nicht durch gr\"o\ss{}ere \"Anderungen ver\"andert werden muss \cite{harrison01}. \\

In einer Document-Store Datenbank ist folgende Hierarchie typischerweise anzutreffen:
\begin{itemize}
\item Ein Dokument ist eine Grundeinheit für das speichern von Daten. Sie ist zu vergleichen mit einer Zeile einer Relationalen Datenbank.
\item Ein Dokument besteht aus einem oder mehreren Key-Value Paaren, und kann weitere Dokumente und Arrays enthalten. die Arrays können wiederum ebenfalls Dokumente enthalten, was eine komplexere Strukturierung ermöglicht.
\item eine Sammlung oder ein Data Bucket ist ein Set von Dokumenten, die einen Zweck teilen. Dies ist äquivalent zu einer relationalen Tabelle. Hier ist jedoch relevant zu erwähnen, dass die Dokumente hier nicht vom gleichen Typ sein müssen, auch wenn es üblich ist, dass solche Dokumente die gleiche Art von Information darstellen \cite{harrison01}.
\end{itemize}

\subsection{Datenmodell}
Im Zentrum des Datenmodells der Document-Stores stehen die Documents, die eine hierarchische Struktur aufweisen. So kann ein Dokument auf viele weitere Dokumente verweisen. Dies ist möglich, da ein Dokument Listen enthalten kann. 
\\

Wie bereits erwähnt ist es möglich mithilfe von Document-Stores ein relationales Schema nachzubilden. Dies ist jedoch häufig nicht sinnvoll, da es größere Vorteile bringt die Dokumente in kleineren Sammlungen zusammenzufassen und in eingebetteten Dokumente die größte Detailstufe zu bieten \cite{harrison01}. 
\\

Sei \ref{fig:doc2} die Darstellung einer relationalen Datenbank mit Filmen und Schauspielern.
\begin{figure}[H]
	\centering
	\includegraphics[scale=0.4]{images/01docstores_2.jpg} 
	\caption{Beispiel eines JSON-Dokuments}\label{fig:doc2}
\end{figure}
In einer relationalen Datenbank würde es eine Tabelle mit Filmen und eine Tabelle mit Schauspielern geben, die zu einer gemeinsamen Tabelle gejoint werden würden um anzuzueigen welcher Schauspieler bei welchem Film mitgewirkt hat. Mithilfe von Document-Stores wäre dieses Schema so auch möglich, jedoch gibt es hierfür einen besseren Ansatz. Da Document-Stores grundsätzlich keine Join-Operationen ermöglichen und Entwickler es bevorzugen wenn die JSON Strukturen sich nah an dem objektorientierten Programmentwurf orientieren, bietet sich ein Entwurf nach \ref{fig:doc1} eher an \cite{harrison01}.
\begin{figure}[H]
	\centering
	\includegraphics[scale=0.3]{images/01docstores_1.jpg}
	\caption{Beispiel eines JSON-Dokuments}\label{fig:doc1}
\end{figure}
Wie in der Abbildung \ref{fig:doc1} zu sehen, befindet sich ein Array von Schauspieler Dokumenten in dem Film Dokument. Dieser Design-Entwurf, der Document-Embedding genannt wird, hat den Vorteil, dass hierdurch auf aufwendige Join-Operationen verzichtet werden kann. Nachteil hierbei ist jedoch, dass die Schauspieler in vielen Dokumenten auftauchen und somit redundant sind. Sollte zudem eines der Schauspieler Attribute verändert werden müssen kann dies zu Inkonsistenzen führen. Ein weiteres Problem kann auftreten, wenn beispielsweise eine unbegrenzte Anzahl an Schauspielern möglich sein soll, da Dokumente in der Regel begrenzt sind. Um dieses Problem zu lösen besteht jedoch die Möglichkeit Indizes zu verwenden \cite{harrison01}.
\begin{figure}[H]
	\centering
	\includegraphics[scale=0.2]{images/01docstores_3.jpg} 
	\caption{Beispiel eines JSON-Dokuments}\label{fig:doc3}
\end{figure}
In Abbildung \ref{fig:doc3} ist zu sehen, wie ein solches System aussehen könnte. 
\\

Grundsätzlich gilt bei modernen Document-Stores, dass Dokumente üblicherweise in JSON bzw. BSON vorliegen \cite{harrison01}.

\subsubsection{JSON}
Die Java-Script-Object-Notation ist ein vom Menschen und Maschinen lesbarer Standard zur Beschreibung von Daten. JSON ist wie XML ein Standard zum Datenaustausch im Web.
In \ref{fig:doc1} ist ein Beispiel für ein JSON-Dokument zu sehen. Vorteil von JSON ist, dass JSON-Dokumente einfach zu zerlegen und umzuwandeln sind. Dies verringert den Aufwand, der in der Anwendungsschicht betrieben werden muss.
\subsubsection{BSON}
Unter BSON sind binär kodierte JSON-Dokumente zu verstehen. BSON erweitert das Modell von JSON um weitere Datentypen und ermöglicht effizientes kodieren und dekodieren innerhalb verschiedener Sprachen.
\subsection{Anwendungsfälle}
JSON basierte Document-Stores haben vor allem in Web-basierten Anwendungen ihre Vorzüge. Hier wird häufig JSON als Datenschicht verwendet, weswegen Document-Stores hier bevorzugt werden.
\\

XML-basierte Document-Stores finden vor allem in Content-Management-Systemen Anwendung, da sie hier ein Management Repository für XML-basierte Textdateien zur Verfügung stellen.
\\

Grundsätzlich lässt sich jedoch feststellen:
\begin{itemize}
\item dass Document-Stores vor allem in Umgebungen florieren, in denen über viele Wege Zugang zu den Daten gewährleistet werden soll, die wiederum vielfältiger Natur sein können. 
\item dass Document-Stores sinnvoll sind, wenn eine große Menge an kleinen, sich wiederholenden Schreib- und Lesevorgängen abgearbeitet werden muss.
\item dass Document-Stores sinnvoll sind, wenn die CRUD-Operationen ohne großartige Join-Operationen umgesetzt werden sollen.
\item dass Document-Stores sinnvoll sind, wenn eine programmiererfreundliche Datenbank umgesetzt werden soll.
\end{itemize}

\subsection{Bewertung}
Hieraus lässt sich evaluieren, dass Document-Stores vielfältige Einsatzmöglichkeiten haben. Vor allem im Bereich der Web-Technologien haben Document-Stores wie MongoDB Stärken, die sie zu guten Alternativen zu klassischen relationalen Datenbanken machen. Durch die Freiheiten bei den zu speichernden Daten und der Einfachheit der Abfragen zeigen sich hier vor allem ihre Stärken. Ferner wird deutlich, dass, gemessen an der Geschwindigkeit in der neue Technologien zur Darstellung von Daten entstehen, Document-Stores im Web die notwendige Freiheit bieten, um angemessen vorzugehen.
