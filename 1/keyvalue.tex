\section{Key-Value Stores}
\subsection{Einleitung}
Key-Value Stores oder Key-Value Datenbanken sind eine der einfachsten Formen der NoSQL Datenbanken. Sie können beliebige Werte (englisch: Values) unter einem definierbaren Key persistieren bzw. laden. Dabei können sie mit einem herkömmlichen relationalen Datenbanksystem (RDBS) verglichen werden in welchem die Tabellen nur zwei Attribute besitzen: eine ID- und eine Value-Spalte \cite{sadalage01}. Im Unterschied zu dieser relationalen Datenbank erwarten Key-Value Stores keinen Datentyp für den Value. Hier können beliebige Werte als Blob gespeichert werden. Auch können Key-Value Datenbanken weder Relationen oder Beziehungen zwischen den Einträgen herstellen. Für die Semantik, Korrektheit der Werte, Beziehungen o.ä., ihre Verwendung, welcher Datentyp vorliegt und etwaiges Fehlerhandling ist allein die Anwendung, welche die Datenbank nutzt verantwortlich. Dieses sehr grundlegende Verhalten soll eine möglichst hohe Performance und Skalierbarkeit ermöglichen. 

\subsection{Eigenschaften}

\subsubsection{Konsistenz}

\subsubsection{Transaktionen}

\subsection{Datenmodell}

\subsection{Abgrenzung zu Document-Stores}
Sehr ähnlich, Unterschied: Das Aggregat muss eine bestimmte Struktur haben, von bestimmtem Typ sein
Mit Document sind weiter Abfragen möglich.. Hier nur per Key alle Daten dazu
In der Praxis verschwimmt die klare Trennung

\subsection{Anwendungsfälle}

\subsection{Bewertung}
