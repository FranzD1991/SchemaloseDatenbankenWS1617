\section{Key-Value Stores}
\subsection{Einleitung}
Key-Value Stores oder Key-Value Datenbanken sind eine der einfachsten Formen der NoSQL Datenbanken. Sie können beliebige Werte (Englisch: Values) unter einem definierbaren Key persistieren und über diesen selektieren. Dabei können sie mit einem herkömmlichen relationalen Datenbanksystem (RDBS) verglichen werden in welchem die Tabellen nur zwei Attribute besitzen: eine ID- und eine Value-Spalte \cite{sadalage01}. Im Unterschied zu dieser relationalen Datenbank erwarten Key-Value Stores keinen Datentyp für den Value. Hier können beliebige Werte als blob gespeichert werden. Auch können Key-Value Datenbanken weder Relationen oder Beziehungen zwischen den Einträgen herstellen. Für die Semantik, Korrektheit der Werte, Beziehungen o.ä., ihre Verwendung, welcher Datentyp vorliegt und etwaiges Fehlerhandling ist allein die Anwendung, welche die Datenbank nutzt, verantwortlich. Dieses sehr grundlegende Verhalten soll eine möglichst hohe Performance und Skalierbarkeit ermöglichen. 

\subsection{Eigenschaften}
Einige Key-Value Stores gehen einen Schritt weiter und können das zu speichernde Aggregat differenzieren. So kann Redis beispielsweise Listen, Sets oder Hashes speichern und unterstützt einfache Operationen auf diesen Datenstrukturen. Somit verschwimmt in der Praxis der Unterschied zwischen Key-Value und Document Stores bei einigen Vertretern sehr.

Im Folgenden werden einige Eigenschaften von Key-Value Stores näher betrachtet und aufgezeigt wie diese sich von einem Standard RDBS unterscheiden.

\subsubsection{Konsistenz}
Da die Key-Value Datenbank Änderungen des Values nicht bestimmen kann, kann die Konsistenz somit nur über den Key und die damit verbundenen Operationen get, put oder delete gewährt werden \cite{sadalage01}. 

In verteilten System wird häufig die \textit{eventually consistent} Eigenschaft angewendet. 

\subsubsection{Transaktionen}
Für Transaktionen gibt es in der NoSQL Welt keine einheitliche Spezifikation. Jeder Anbieter von Key-Value Stores implementiert diese anders, Riak benutzt hier beispielsweise das Quorum-Konzept \cite{sadalage01}.

\subsubsection{Query-Eigenschaften}
Da sich Key-Value Datenbanken nicht für den gespeicherten Value interessieren kann man Queries nur auf den Key ausführen. Auf Grund dieses sehr einfachen Handlings sind Key-Value Stores nicht für Anwendungen geeignet, welche über den Inhalt filtern oder suchen müssen. Hier muss die Anwendung wissen welche Daten unter welchem Key zu finden sind und welchen Key sie aufrufen möchte. Da nicht immer der Fall eintritt, dass der Key zu einem Datum bekannt ist, stellen manche Key-Value Datenbanken allerdings eine Suchfunktionalität oder die Möglichkeit der Indexierung bereit, damit nicht jeder Datensatz von der Anwendung geladen und einzeln durchsucht werden muss \cite{sadalage01}. 

\subsubsection{Struktur der Daten}
Wie bereits erwähnt können in Key-Value Datenbanken verschiedenste Datentypen wie blob, text, JSON, XML oder weitere gespeichert werden, da die Datenbank sich um diese Werte nicht weiter kümmert. Bestimmte Datenbanken, wie beispielsweise Riak können einem Request einen Content-Type übergeben womit diese weiß wie sie die Daten behandeln soll \cite{sadalage01}.

\subsubsection{Skalierbarkeit}
Um die Performance zu erhöhen arbeiten die meisten Key-Value Stores mit „Sharding“. Hierbei entscheidet der Wert des Keys auf welchem Knoten bzw. Shard der Datenbank ein Datum gespeichert wird. So werden beispielsweise alle Keys, welche mit g beginnen auf einem anderen Shard gespeichert als solche, die mit a oder einem nummerischen Wert beginnen. Zusätzliche Shards können sehr einfach ergänzt werden. Um hierbei die Konsistenz nicht zu gefährden, sollte ein Knoten ausfallen, können diese Shards repliziert werden. Die einzelnen Key-Value Datenbanken bieten hierbei verschiedene Möglichkeiten an. So kann bei Riak angegeben werden, wie viele Knoten bei Schreib- und Lesezugriffen mindestens beteiligt sein müssen damit die Transaktion als erfolgreich gilt. 

\subsection{Abgrenzung zu Document-Stores}
Sehr ähnlich, Unterschied: Das Aggregat muss eine bestimmte Struktur haben/ von bestimmtem Typ sein
Mit Document Stores sind Abfragen auf Attribute möglich. Hier nur per Key.
In der Praxis verschwimmt die klare Trennung.
Unfertig …

\subsection{Anwendungsfälle}
Da die Keys in den Key-Value Stores von der Anwendung verwaltet werden, können diese verschiedenen Charakter haben. Eine Möglichkeit ist diese Keys durch einen bestimmten Algorithmus, zum Beispiel eine Hashfunktion, zu generieren. Man kann den Key auch durch Benutzerinformationen, wie Email-Adresse oder User-ID generieren. Alternativ kann dieser auch aus einem Zeitstempel oder anderen Attributen erstellt werden. Dies macht Key-Value Stores attraktiv um Session Informationen, Einkaufswagen im Online-Shopping, Benutzerprofile oder Protokolldateien abzuspeichern, da diese Daten unter einem eindeutigen Key als ein Objekt gespeichert werden \cite{sadalage01}. 

Anwendungsfälle in welchen Key-Value Stores weniger gut geeignet sind:
\begin{itemize}
\item bei Abhängigkeiten oder Beziehungen zwischen den Daten wie sie in herkömmlichen RDBS zu finden sind.
\item bei \textit{Multioperation Transactions}, in welchen bei einem Fehler die restlichen Operationen zurückgedreht werden müssten.
\item bei (Such-)Anfragen, welche den Inhalt der Daten betreffen.
\item bei Operationen, die sich auf mehrere Datensätze zum gleichen Zeitpunkt beziehen, da eine Operation in einem Key-Value Store auf einen Key beschränkt ist.
\end{itemize}

\subsection{Bewertung}
Key-Value Stores ziehen ihren großen Vorteil aus der Einfachheit ihrer Implementierung. Diese ermöglicht ihnen eine hohe Performance. Hierbei verzichten sie absichtlich auf Funktionen, wie Beziehungen, feste Datentypen und die Möglichkeit Abfragen auf jedes Attribut eines Datensatzes zu stellen, die durch die relationale Datenbankwelt bekannt geworden sind. 