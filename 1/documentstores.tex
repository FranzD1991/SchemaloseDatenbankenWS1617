\section{Document Stores}
\subsection{Einleitung}
Document-Stores speichern strukturierte Daten in Datensätzen, die Dokumente genannt werden. Grundsätzlich vereinigen sie die Schemafreiheit von Key-Value-Stores mit der Möglichkeit zur Strukturierung der gespeicherten Daten. Somit lassen sich Document-Stores als Erweiterung von Key-Value Stores betrachten. 
//
\subsection{Eigenschaften}
Bei der Betrachtung von Document-Stores bezieht sich das Wort "document" auf die hierarchische Struktur der abgespeicherten Daten. Die hier betrachteten Dokumente bestehen jedoch lediglich aus binären Daten oder einfachem Text. Ferner kann es sich um Semi-strukturierte Daten wie JSON oder XML handeln. 
//


\subsection{Datenmodell}
\subsection{Abgrenzung JSON basierter Document Stores gegenüber XML basierter Document Stores}
\subsection{Anwendungsfälle}
\subsection{Bewertung}
