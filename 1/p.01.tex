\chapter{Vorwort}
\cite{edlich01}
\cite{sadalage01}
\cite{redmont01}
\cite{Jing01}
\cite{Bogdan01}
\cite{Meier01}
\cite{Fasel01}
\cite{Meier02}
\cite{Fasel02}
\cite{Boicea01}
\cite{Parker01}
\cite{Abramova01}
\cite{Faraj01}
\cite{Shakuntala01}
\cite{Hows01}
\cite{fowler01}
\cite{harrison01}
Zuordnungen von Aufgabenpaketen auf die Gruppenmitglieder. Diese Zuordnung fließt insbesondere dann in die Einzelbewertung mit ein, wenn eine homogene Gruppenleistung nur schwer festgestellt werden kann.
In der eidesstattlichen Erklärung (s.u.) gibt es einen Verweis auf das Vorwort.
Die Zuordnung kann tabellarisch erfolgen. Damit die Tabelle nicht zu breit wird, definieren Sie einfach einen DreiLetterCode pro Gruppenmitglied
Für die folgenden (falls Vorhanden) Abschnitte (Zeilen) ...

    Ausgewählte technologische Grundlagen
    Ausgewählte Persistenzmodelle
    Ausgewählte Persistenzsysteme und Big Data Frameworks
        Vorstellung des Systems
        Installation des Systems
        Ad-Hoc-Anwendung des Systems (CRUD)
        Implementierung der Beispiel-Datenbank
    Ausgewählte Anwendungsszenarien
        Spezifikation des Anwendungsszenarios
        Spezifikation der konkreten Anwendung
        Implementierung der Anwendung

... werden jeweils die folgenden Zuordnungen (Spalten) erwartet:

    Hauptverantwortlicher schriftliche Ausarbeitung
    Hauptverantwortlicher praktische Umsetzung am Rechner
    Hauptverantwortlicher Erstellung Vortrag
    Hauptverantwortlicher Vortrag vorgetragen