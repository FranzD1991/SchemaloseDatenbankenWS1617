\chapter{Anhänge}
\section{Use Cases}\label{use_cases}

\paragraph{Use Case 1} Als Student möchte ich herausfinden welche Professoren in der Vergangenheit Themen unterstützt haben, die mich interessieren. \\
\emph{Input:} Thema das mich interessiert (als String). \\
\emph{Output:} Liste mit Professoren die im Zusammenhang mit dem Thema 1., 2. oder 3. Prüfer waren. \\
\emph{Query:}

\begin{lstlisting}[caption={Query zu Use Case 1},language=java,captionpos=t,numbers=none, numberstyle=\tiny,breaklines=true]
db.getCollection('abDBsimpleCSV')
.distinct('1_Pruefer',{"Thema_der_Arbeit":/.*Thema_der_Arbeit.*/})
\end{lstlisting}\label{lst:query1}

\paragraph{Use Case 2} Als Student möchte ich wissen, welche Professoren Bachelor- und Masterthesen anbieten oder angeboten haben. \\
\emph{Output:} Liste von allen Professoren. \\
\emph{Query:}

\begin{lstlisting}[caption={Query zu Use Case 2},language=java,captionpos=t,numbers=none, numberstyle=\tiny,breaklines=true]
db.getCollection('abDBsimpleCSV').distinct('1_Pruefer',{$or:[{"1_Pruefer":{$ne:""}},{"2_Pruefer":{$ne:""}},{"3_Pruefer":{$ne:""}}]})
\end{lstlisting}\label{lst:query2}

\paragraph{Use Case 3} Als Student möchte wissen welche Themen in der Vergangenheit von einem bestimmten Professor angeboten wurden. \\
\emph{Input:} Professorname. \\
\emph{Output:} Liste mit Themen bei welchem der Professor Prüfer war. \\
\emph{Query:}

\begin{lstlisting}[caption={Query zu Use Case 3},language=java,captionpos=t,numbers=none, numberstyle=\tiny,breaklines=true]
db.getCollection('abDBsimpleCSV').distinct('Thema_der_Arbeit',{$or:[{"1_Pruefer":"Professorname"},{"2_Pruefer":"Professorname"},{"3_Pruefer":"Professorname"}]})
\end{lstlisting}\label{lst:query3}

\paragraph{Use Case 4} Als Professor möchte ich mir die Anzahl der der erfolgreichen Abschlüsse für ein bestimmtes Semester ausgeben lassen. \\
\emph{Input:} Semester. \\
\emph{Output:} Anzahl der Abschlüsse. \\
\emph{Query:}

\begin{lstlisting}[caption={Query zu Use Case 4},language=java,captionpos=t,numbers=none, numberstyle=\tiny,breaklines=true]
db.abDBsimpleCSV.find({$and:[{Semester: "(Semstername)", Zeugnis: 1}]}).count()
\end{lstlisting}\label{lst:query4}

\paragraph{Use Case 5} Als Student möchte ich wissen, welche Unternehmen Bachelor- und Masterthesen betreut haben. \\
\emph{Output:} Liste aller Unternehmen. \\
\emph{Query:}

\begin{lstlisting}[caption={Query zu Use Case 5},language=java,captionpos=t,numbers=none, numberstyle=\tiny,breaklines=true]
db.abDBsimpleCSV.distinct('Firma',{Firma: {$ne:""}})
\end{lstlisting}\label{lst:query5}

\paragraph{Use Case 6} Als Professor oder als Student möchte ich mir anzeigen lassen, welcher Professor bisher die meisten Abschlussarbeiten unterstützt hat. \\
\emph{Output:} Liste mit Professoren und der Anzahl an unterstützten Abschlussarbeiten. \\
\emph{Query:}

\begin{lstlisting}[caption={Query zu Use Case 6},language=java,captionpos=t,numbers=none, numberstyle=\tiny,breaklines=true]
db.abDBsimpleCSV.aggregate([{$match:{"1_Pruefer": {$not: {$size: 0}}}},{$unwind: '$1_Pruefer'}, {$group: {_id: '$1_Pruefer', Anzahl: {$sum: 1}}}, {$match: {Anzahl: {$gte: 2}}},{$sort: {Anzahl: -1}} ,{$limit:200}])
\end{lstlisting}\label{lst:query6}

\paragraph{Use Case 7} Als Professor möchte ich wissen, welche Themen bereits vorgeschlagen oder bearbeitet worden sind. \\
\emph{Input:} Thema. \\
\emph{Output:} Liste aller Themen die dem Input entsprechen. \\
\emph{Query:}

\begin{lstlisting}[caption={Query zu Use Case 7},language=java,captionpos=t,numbers=none, numberstyle=\tiny,breaklines=true]
db.abDBsimpleCSV.distinct('Thema_der_Arbeit',{"Thema_der_Arbeit":/.*(Thema).*/})
\end{lstlisting}\label{lst:query7}

\paragraph{Use Case 8} Welche Professoren haben wieviele Arbeiten zu einem Thema betreut. \\
\emph{Input:} Stichwort zum Thema (Such-String). \\
\emph{Output:} Anzahl Themen mit Such-String gruppiert nach Name des Profs, sortiert nach Anzahl Themen. \\
\emph{Query:}

\begin{lstlisting}[caption={Query zu Use Case 8},language=java,captionpos=t,numbers=none, numberstyle=\tiny,breaklines=true]
db.abDBsimpleCSV.aggregate([
{$match:{"Thema_der_Arbeit":/.*datenbank*/}},
{$match:{"1_Pruefer": {$not: {$size: 0}}}},
{$unwind: '$1_Pruefer'},
{$group: {_id: '$1_Pruefer', Anzahl: {$sum: 1}}},
{$sort: {Anzahl: -1}} ,{$limit:200}])
\end{lstlisting}\label{lst:query8}